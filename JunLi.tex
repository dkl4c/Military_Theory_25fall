\documentclass[12pt]{ctexart}
\usepackage[margin=2.5cm]{geometry}
\usepackage{xcolor}
\usepackage{fontspec}

\title{军事理论思考题}
\author{局部紧豪斯多夫}
\date{\today}

\begin{document}
\large

\maketitle

\section{第一周\ 军事高科技与现代战争}
\subsection{军事高科技的概念}
建立在现代科学技术成就基础上,处于当代科学技术前沿,以\textbf{\textcolor{red}{信息智能科技}}为核心,
在\textbf{\textcolor{red}{军事领域}}发展和应用,
对\textbf{\textcolor{red}{武器装备发展}}和\textbf{\textcolor{red}{现代战争演变}}起巨大作用的高科技总称
\subsection{军事高科技的主要特点}
高智力,高投资,高竞争,高风险,高收益,高保密,高速度
\subsection{美国三次“抵消战略”的对象及内容}
第一次:华约,核武器

第二次:苏联,常规武器

第三次:中国,“全球监视 —— 精确打击”的联合作战体系
\subsection{新时代我军武器装备的地位作用}
军队现代化的重要标志

军事斗争准备的重要基础

国家安全和民族复兴的重要支撑

国际战略博弈的重要筹码

\section{第二周\ 国防动员}
\subsection{国防的概念是什么?}
国家为防备和抵抗侵略,制止武装颠覆和分裂,保卫国家的主权、统一、领土完整、安全和发展利益所进行的军事活动,
以及与军事有关的政治、经济、外交、科技、文化、教育等方面的活动。
\subsection{国防动员的概念是什么?}
国防动员,是指国家为应对战争或其他威胁,
采取非常措施将社会诸领域资源由平时状态转入战时状态或紧急状态的一系列活动。
\subsection{国防动员的功能作用有哪些?}
国防动员的功能作用可概括为:战时应战、急时应急、平时服务。
\subsection{国防动员的主要内容包括哪些?}
国防动员的主要内容包括武装力量动员、国民经济动员、人民防空动员、交通战备动员和政治动员等。
\subsection{怎么理解武装力量动员?}
武装力量动员,是指为了应对战争突发事件或紧急状态的需要,
国家有计划、有组织地将军队和其他武装组织由平时状态转入战时状态所进行的一系列活动。
\subsection{怎么理解人民防空动员?}
人民防空动员,是指国家发动和组织人民群众采取一系列措施,防范敌方空中袭击并减轻空袭后果的活动。

\section{第三周\ 美国军事基本情况}
\subsection{美军有哪几个军种?}
海军、陆军、空军、海军陆战队、海岸警卫队、太空军。
\subsection{美军有哪几个战区司令部和职能司令部?}
6个战区司令部:北方司令部、南方司令部、非洲司令部、欧洲司令部、中央司令部、印太司令部。

5个职能司令部:战略司令部、军事运输司令部、特种作战司令部、网络司令部、太空司令部。

\section{第五周\ 中国古代军事思想}
\subsection{中国古代军事思想的发展历程可以分为哪几个阶段?}
第一阶段:初见蓓蕾夏商周(\textbf{\textcolor{red}{形成}})

第二阶段:春秋战国渐成熟(\textbf{\textcolor{red}{成熟}})

第三阶段:发展时逢秦五代(\textbf{\textcolor{red}{发展}})

第四阶段:自成体系宋明清(\textbf{\textcolor{red}{体系}})
\subsection{中国古代军事思想的体系阶段是哪个时期,重要标志是什么?}
宋元明清(前期);宋朝《武经七书》。
\subsection{《孙子兵法》一共有多少篇,第一篇和最后一篇的篇名是什么?}
13篇;《计》;《用间》。
\subsection{《孙子兵法》提出的为将5条标准是什么?}
智、信、仁、勇、严。

\section{第六周\ 智能化战争研究}
\subsection{智能化战争的概念}
智能化战争,是基于\textbf{\textcolor{red}{物联}}网络信息系统,
运用\textbf{\textcolor{red}{智能化武器装备}}及相应\textbf{\textcolor{red}{作战方法}},
在\textbf{\textcolor{red}{陆、海、空、天、电、网}}及\textbf{\textcolor{red}{认知领域}}进行的一体化战争
\subsection{智能化战争的主要特点}
战争手段混合化,战场环境物联化,力量体系无人化,对抗方式群集化,指挥决策云脑化
\subsection{应对智能化战争的思考}
一是全面更新智能化\textbf{\textcolor{red}{战争观念}},二是科学组建智能化\textbf{\textcolor{red}{军事力量}},
三是创新发展\textbf{\textcolor{red}{人民战争战略战术}}

\section{第七周\ 新军事革命}
\subsection{什么是新军事革命?其主要标志是什么?}
世界新军事革命,是20世纪末期以来在世界范围内发生的一场以信息化为核心的全面而深刻的革命,
包括军事技术、武器装备、体制编制、战争形态、军事理论、作战方式、军事训练等诸多方面的根本性变革。

主要标志:信息化武器据系统逐渐主宰战场,出现知识密集型的信息化军队,一体化联合作战成为基本作战形式,
军事理论及其体系彻底革新,信息化战争最终取代机械化战争。
\subsection{新军事革命的发展演变有哪些阶段?}
起步阶段(20世纪70年代—90年代)

展开阶段(20世纪90年代—21世纪10年代)

完成阶段(2010年—21世纪下半叶)

\subsection{新军事革命的主要内容是什么?}
军事技术的根本性变革

武器装备的根本性变革

军队体制编制的根本性变革

军事理论的根本性变革
\subsection{什么是中国特色军事变革?其主要内容是什么?}
中国特色军事变革,是指为应对当代世界新军事革命挑战,从中国国情、军情出发,
在军事领域实行的以信息化为本质特征和核心内容的全面性、系统性的革新。

主要内容:

发展方向:信息化、智能化

基本目标:打赢信息化智能化战争

基本途径:机械化信息化智能化融合发展

发展步骤:与国家发展战略相适应

根本动力:改革创新

根本保证:坚持中国共产党的绝对领导
\subsection{国防和军队现代化建设新“三步走”战略是什么?}
2027实现建军百年目标\\
加快推进军事理论现代化、军队组织形态现代化、军事人员现代化、武器装备现代化,确保掌握捍卫国家主权、安全、发展利益的战略主动。

2035基本实现国防和军队现代化\\
机城化高度发达,信息化基本实现,智能化取得重大进展,为把我军全面建成世界一流军队奠定坚实基础。

2050把人民军队全面建成世界一流军队。

\subsection{随着国防和军队改革的深化,我军组织体制发生了哪些重大变化?}
此次国防和军队改革,是一次结构性、革命性的体制重塑。打破了长期实行的总部体制、
大军区体制、大陆军体制,形成“军委管总、战区主战、军种主建”新格局。

裁减军队员额30万。大幅压减陆军现役员额,适度增加海军、火箭军现役员额,
优化各军兵种内部结构。优化后备力量结构。调整作战力量部署,形成与维护新时代国家安全需要相适应的战略布局。

\section{第八周\ 征兵优抚安置政策}
\subsection{经济补助或补偿}
2年义务兵期间,本科学生不低于31.43万元各类补助和补偿。

\subsection{学业支持}
学生休学期间应征入伍的,可以办理服兵役休学延期;

学生试读期间应征入伍的,可以办理服兵役休学,试读期顺延。


申请延长学习年限。在校本科学生或本科新生应征入伍、退役后复学的,
在校学习年限可在专业学制基础上适当延长,但在校学习总年限不得超过六年。
\subsection{升学优待}
参加全国硕士研究生招生考试,
初试总分加10分,同等条件下优先录取。


“退役大学生士兵”专项硕士研究生招生。
\subsection{就业落户}
\subsubsection{定向招聘政策}
公务员考录、事业单位、国有企业、非公经济组织招聘,数量分别不低于当年列入人员范围退役大学生士兵人数的10\%、15\%、15\%和10\%。
\subsubsection{北京落户}
非京籍大学本科学历以上,从本市入伍的退役士兵,被本市用人单位接收的,可以落户北京。

\section{第九周\ 中国武装力量}
\subsection{中国武装力量包括:}
中国人民解放军、中国人民武装警察部队、中国民兵。

\subsection{中国人民解放军现役部队包括:}
陆军、海军、空军、火箭军、军事航天部队、网络空间部队、信息支援部队、联勤保障部队。

\subsection{中国人民解放军陆军成立的时间是:}
1927年8月1日。
\subsection{中国人民解放军海军成立的时间是:}
1949年4月23日。
\subsection{中国人民解放军空军成立的时间是:}
1949年11月11日。

\subsection{中国人民解放军陆军、海军、空军、火箭军的发展战略是什么?}
陆军:机动作战、立体攻防;

海军:近海防御、远海防卫;

空军:空天一体、攻防兼备;

火箭军:核常兼备、全域慑战。


\section{第十周\ 核生化武器威胁及其防护}
\subsection{核武器的杀伤破坏因素有哪些?}
光辐射、冲击波、早期核辐射、核电磁脉冲、放射性沾染。

\subsection{生物武器的危害主要有哪些特点?}
致病性强,杀伤程度严重;污染和杀伤范围广;具有传染性,难防难治;无即时杀伤作用,危害时间长。

\subsection{化学武器按其毒害机理和作用,可以分为哪几类?}
神经性毒剂、糜烂性毒剂、全身中毒性毒剂、窒息性毒剂、失能性毒剂、刺痛性毒剂。

\subsection{我国面临的核生化威胁主要包括哪些?}
军事强国拥有强大的核化生武器库;

军事强国推进威慑战略;

国际公约难以制止核化生武器的发展;

核化生次生危害严重。


\section{第十一周\ 毛泽东军事思想}
\subsection{什么是毛泽东军事思想?}
毛泽东军事思想,是以毛泽东为代表的中国共产党人关于中国革命战争和军队问题的科学理论体系。
\subsection{什么是人民战争?}
人民战争,是指广大人民群众为反抗阶级压迫或外敌入侵而组织和武装起来进行的战争。
\subsection{政治工作的三大原则}
官兵一致、军民一致、瓦解敌军。
\subsection{毛泽东军事思想的主要内容}
无产阶级的战争观和方法论、人民军队建设理论、人民战争思想、人民战争的战略战术、国防建设理论。
\subsection{毛泽东军事思想的历史地位}
毛泽东军事思想对马列主义军事理论作出了重大而独特的贡献;

毛泽东军事思想在世界上具有广泛而深刻的影响;

毛泽东军事思想永远是我军克敌制胜的法宝。

\section{第十二周\ 信息化作战平台}
\subsection{什么是信息化作战平台?}
信息化作战平台是以信息化武器装备系统为核心,具有运载、投送和管理控制功能并可以作为武器依托的载体部分。
\subsection{信息化作战平台通常分为哪几类?}
信息化作战平台通常分为陆上信息化作战平台、海上信息化作战平台、
空中信息化作战平台、太空(空间)信息化作战平台、无人作战平台。
\subsection{什么是陆上信息化作战平台?}
陆上信息化作战平台是指大量采用信息技术的各类坦克、装甲车辆、自行火炮和导弹发射装置等。
\subsection{海上信息化作战平台包括哪些?}
海上信息化作战平台包括:
 
水面:航空母舰、巡洋舰、驱逐舰、护卫舰、两栖登陆舰、导弹艇、扫雷艇

水下:常规潜艇和核动力潜艇
\subsection{什么是空间信息化作战平台?}
空间信息化作战平台是指能实施各类太空信息支援、指挥,
以及能对地方卫星和空中、海上、陆地目标实施攻击的各类太空平台。
\subsection{榴弹炮与加农炮的主要区别?}
榴弹炮身管较短、初速较小、射程较近;

加农炮身管较长、初速较大、射程较远。
\subsection{潜艇如何分类?}
按动力性质,可分为常规潜艇和核潜艇。

其中,核潜艇按照功能可分为弹道导弹核潜艇和攻击型核潜艇。
\subsection{信息化作战平台发展趋势主要有哪些?}
隐身化;无人化;智能化。

\section{第十三周\ 日本军事基本情况}
\subsection{日本武装力量主要由哪几部分组成?}
主要由现役部队、文职人员和预备部队三部分组成。
\subsection{日本现役部队由哪几部分组成?}
由陆上自卫队、海上自卫队和航空自卫队三部分组成。
\subsection{二战结束后,日本在1970年首次自主提出本国军事战略,名称是什么?}
日本二战后于1970年首次自主提出本国军事战略,名称为专守防卫。
\subsection{日本现行军事战略的名称是什么?有什么含义?其实质是什么?}
多域联合防卫,该战略包含三层含义:一确保日本的和平与安全;二维护日本的国家利益;三改善印太及全球安全保障环境。
多域联合防卫的实质是一种攻势防卫。
\subsection{日本政府标榜的“无核三原则”是指什么?}
是指不制造、不拥有、不运进核武器。
\subsection{日本强化日美军事同盟的主要战略意图是什么?}
一是依靠日美同盟,确保日本自身的安全;二是依托日美同盟,介入和遏制地区危机;三是借助日美同盟,主导全球安全事务谋取更大利益。
\subsection{日本现行军事战略的发展走向对我国安全和利益有哪些重大影响?}
一是牵制我解决台湾问题,影响我祖国统一大业;二是威胁我海上安全和海洋利益,压缩中国海疆战略缓冲;三是竭力对我实施战略围堵,阻遏我国军力发展强大。


\section{第十四周\ 中国国防}
\subsection{国防的定义}
国家为防备和抵抗侵略,制止武装颠覆和分裂,
保卫国家的主权、统一、领土完整、安全和发展利益所进行的军事活动,
以及与军事有关的政治、经济、外交、科技、文化、教育等方面的活动。
\subsection{国防类型}
扩张型、自卫型、联盟型、中立型。

\section{第十五周\ 国家安全形势}
\subsection{“十五五”时期我国安全发展面临的形势?}
我国发展处于战略机遇和风险挑战并存、不确定难预料因素增多的时期。
\subsection{我国周边安全环境的总体特点?}
“三多两乱”,邻国多、强国多、人口多,经济状态发展乱、国家之间关系乱。
\subsection{新安全格局的核心内涵?}
传统安全与非传统安全并重,自身安全与共同安全兼顾,维护安全和塑造安全统一。

\section{第十六周\ 习近平强军思想}
\subsection{简述党在新时代的强军目标主要内容}
建设一支听党指挥、能打胜仗、作风优良的人民军队,到2027年实现建军一百年奋斗目标,
到2035年基本实现国防和军队现代化,到本世纪中叶把人民军队建设成为世界一流军队。
\subsection{简述当代中国马克思主义的军事观和方法论的主要内容}
坚持政治引领,坚持以武止戈,坚持积极进取,坚持统筹兼顾,坚持敢打必胜。 

\subsection{简述什么是中央军委主席负责制}
中央军委主席负责中央军委全面工作,领导指挥全国武装力量,决定国防和军队建设一切重大问题。

\subsection{依法治军战略的主要内容是什么?}
坚持党对军队绝对领导,坚持战斗力标准,坚持建设中国特色军事法治体系,
坚持按照法治要求转变治军方式,坚持从严治军铁律,坚持抓住领导干部这个“关键少数”,
坚持官兵主体地位,坚持贯彻全面依法治国要求。

\subsection{简述新时代政治建军方略的主要内容}
明确政治建军是人民军队立军之本;

明确政治工作永远是我军的生命线;

明确政治整训要突出政治上的正本清源;

明确掌握思想领导是掌握一切领导的基础;

明确党的力量来自组织、部队凝聚力战斗力来自组织;

明确枪杆子要始终掌握在对党忠诚可靠的人手中;

明确严才能正纲纪、严才能肃军威、严才能出战斗力;

明确军中绝不能有腐败分子藏身之地;

明确作风优良才能塑造英雄部队;

明确军政军民团结是我军胜利法宝。

\subsection{简述习近平强军思想战略布局的主要内容}
政治建军、改革强军、科技强军、人才强军、依法治军。

\end{document}
